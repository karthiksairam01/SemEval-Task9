% This must be in the first 5 lines to tell arXiv to use pdfLaTeX, which is strongly recommended.
\pdfoutput=1
% In particular, the hyperref package requires pdfLaTeX in order to break URLs across lines.

\documentclass[11pt]{article}

% Change "review" to "final" to generate the final (sometimes called camera-ready) version.
% Change to "preprint" to generate a non-anonymous version with page numbers.
\usepackage[preprint]{acl}

% Standard package includes
\usepackage{times}
\usepackage{latexsym}

% For proper rendering and hyphenation of words containing Latin characters (including in bib files)
\usepackage[T1]{fontenc}
% For Vietnamese characters
% \usepackage[T5]{fontenc}
% See https://www.latex-project.org/help/documentation/encguide.pdf for other character sets

% This assumes your files are encoded as UTF8
\usepackage[utf8]{inputenc}

% This is not strictly necessary, and may be commented out,
% but it will improve the layout of the manuscript,
% and will typically save some space.
\usepackage{microtype}

% This is also not strictly necessary, and may be commented out.
% However, it will improve the aesthetics of text in
% the typewriter font.
\usepackage{inconsolata}

%Including images in your LaTeX document requires adding
%additional package(s)
\usepackage{graphicx}

% If the title and author information does not fit in the area allocated, uncomment the following
%
%\setlength\titlebox{<dim>}
%
% and set <dim> to something 5cm or larger.

\title{Towards Explainable Food Hazard Detection: A Neuro-Symbolic Approach}


% Author information can be set in various styles:
% For several authors from the same institution:
% \author{Author 1 \and ... \and Author nhmukh \\
%         Address line \\ ... \\ Address line}
% if the names do not fit well on one line use
%         Author 1 \\ {\bf Author 2} \\ ... \\ {\bf Author n} \\
% For authors from different institutions:
% \author{Author 1 \\ Address line \\  ... \\ Address line
%         \And  ... \And
%         Author n \\ Address line \\ ... \\ Address line}
% To start a separate ``row'' of authors use \AND, as in
% \author{Author 1 \\ Address line \\  ... \\ Address line
%         \AND
%         Author 2 \\ Address line \\ ... \\ Address line \And
%         Author 3 \\ Address line \\ ... \\ Address line}

\author{Neelima Prasad \\
  CU Boulder\\
  \texttt{neelima.prasad@colorado.edu} \\\And
  Karthik Sairam \\
  CU Boulder\\
  \texttt{email@domain} \\\And
  Advait Deshmukh \\
  CU Boulder\\
  \texttt{email@domain} \\}
  

\begin{document}
\maketitle


\section{Introduction}

\paragraph{Task / Research Question Description}
Food safety is becoming an increasingly important issue worldwide. As our food systems grow more complex and interconnected, the risks of contamination and food borne illnesses rise. Moreover, with the rise of social media, there are a myriad of food safety reports flooding the web that is difficult to sort through. We propose a classification model that would be able to identify and explain food-related risks as well as particular hazards from online sources. For any text source, our model would be able to perform text classification for food hazard prediction to the type of hazard and product, as well as food hazard and product vector detection, to predict the exact hazard and product.

\paragraph{Motivation and Limitations of existing work}
Food risk classification based on texts is currently underexplored. Previous methods [ CITE] struggle with explainability, since there are many diverse approaches for it. Due to the potential high economic impact, transparency is crucial for this task. We aim to create a high explainable model that can not only predict hazards but is also understandable. This transparency is crucial for trust and practical application in food safety. Given the need for explainability with this task, we turn to a neuro-symbolic approach to leverage human reasoning and discretization which results in a more interpretable model. 

\paragraph{Likely challenges and Mitigations}
This task is challenging in general as it involves balancing the model's ability to generalize across related tasks while specializing in a particular domain. The main challenge to overcome in solving this task would be to create an architecture to improve upon the baseline that BERT provides, which is an F1 score of 0.83. In class, we have studied techniques that leverage modular approaches to improve upon neural methods. Our biggest struggle will be to exploit domain specific knowledge in a way that will boost performance and generate results that are more explainable and generalisable. If our experiments do not achieve the results that we desire, we will simply present our architecture and offer some insights as to what information our model did capture and what information it failed to learn.  

\section{Related Works}

% 3 sentences

\cite{10.1145/3529755} present a survey on the explainability of deep models in NLP by underlining the importance of explainability in domains where understanding the decision-making process is critical, which directly corresponds to our task of building an explainable food hazard detection model. The authors focus on elucidating why explainability is especially tricky when it comes to textual data, supporting their claim with reasons such as the opaque nature of word embeddings and the inherent interpretability of the attention mechanism in transformers. The specific avenue that we would like to consider building upon from this paper is that of quantitatively assessing the explainability of our model and its textual data.


\paragraph{Generating explanations that are model specific}

% 4 sentences

The authors of \cite{assael2022restoring} propose Ithaca, a deep neural network architecture that can perform the tasks of restoring ancient texts from Greek inscriptions, in addition to also attributing a place of origin and date of writing of the inscription. We think that this is relevant because of its model-specific explanation, in which the authors' claim that a "high-level of generalization is often involved" (in epigraphy), resonates with the fact that food risk classification often is not accompanied by transparency. Their use of models that perfectly fit the three tasks Ithaca excels at (text restoration, geographical attribution and geographical attribution), suggests that such an approach intends to enable the readers and the authors to not only have a deeper understanding about the solution proposed, but also to reason about the structure of the model. We will be building on a similar approach, albeit incorporating a symbolic approach for the food hazard detection solution, where we will leverage model-specific explanations.

\paragraph{Generating explanations that are model agnostic}

% 4 sentences

In \cite{ribeiro-etal-2016-trust}, the main focus is to develop a system that explains why a classifier made a certain prediction, which is done by identifying the important parts of the input that contribute towards the decision making and presenting the visual artifacts that establish a relationship between the input and the prediction. Since model-agnostic methods do not take into account the model's structure since they work on a black-box approach, the authors intend to develop Local Interpretable Model-agnostic Explanations (LIME). The outlined drawbacks of evaluating only the accuracy of the model (dataset shift, cross validation overestimation and data leakage) to explain it's performance are interesting, since they can also be applied directly towards the explainable food hazard detection model we will be working on. Our proposed methodology takes a slightly different approach, which is in being explainable right from the start, hence not needing an approach to generate explanations using a separate methods like the one the authors have proposed.


% Bibliography entries for the entire Anthology, followed by custom entries
%\bibliography{anthology,custom}
% Custom bibliography entries only






\nocite{*}






% Bibliography entries for the entire Anthology, followed by custom entries
%\bibliography{anthology,custom}
% Custom bibliography entries only
\newpage
\bibliography{custom}




\end{document}





\end{document}

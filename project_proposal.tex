\documentclass{article}
\usepackage{graphicx, hyperref} % Required for inserting images

\title{Towards Explainable Food Hazard Detection: A Neuro-Symbolic Approach}

\author{Neelima Prasad, Karthik Sairam, Advait Deshmukh}
\date{September 20, 2024}

\begin{document}

\maketitle

\section{Introduction}
\paragraph{Task / Research Question Description}
Food safety is becoming an increasingly important issue worldwide. As our food systems grow more complex and interconnected, the risks of contamination and food borne illnesses rise. Moreover, with the rise of social media, there are a myriad of food safety reports flooding the web that is difficult to sort through. We propose a classification model that would be able to identify and explain food-related risks as well as particular hazards from online sources. For each text source, our model would be able to perform text classification for food hazard prediction to the type of hazard and product, as well as food hazard and product vector detection, to predict the exact hazard and product.

\paragraph{Motivation and Limitations of existing work}
Food risk classification based on texts is currently underexplored. Previous methods [ CITE] struggle with explainability, since there are many diverse approaches for it. Due to the potential high economic impact, transparency is crucial for this task. We aim to create a high explainable model that can not only predict hazards but is also understandable. This transparency is crucial for trust and practical application in food safety. Given the need for explainability with this task, we turn to a neuro-symbolic approach to leverage human reasoning and discretization which results in a more interpretable model. 

\paragraph{Likely challenges and mitigations}

\section{Related Works}



\end{document}